\title{\textbf{UEBS Doctoral Programme Strategy}}
\documentclass[11pt,a4paper,]{article}
\usepackage[utf8]{inputenc}
\usepackage[hyphens]{url}
\usepackage{graphicx}
\date{\vspace{-5ex}}
\usepackage{hyperref}
\usepackage{xcolor}
\usepackage{graphicx}
\usepackage{placeins}
\usepackage{float}
\usepackage{subfigure}
\hypersetup{
	colorlinks,
	linkcolor={blue!50!black},
	citecolor={blue!50!black},
	urlcolor={blue!80!black}
}
\usepackage[flushmargin]{footmisc}
\usepackage[scale=0.70,tmargin=2cm, bmargin=2cm, lmargin=3cm, rmargin=3cm,footnotesep=1cm,]{geometry}

\begin{document}
\maketitle 
\section*{Vision}
To have an outstanding global doctoral programme that trains scholars to conduct rigorous, interesting, challenging, innovative and impactful research that contribute significantly to knowledge, policy and practice. 

\section*{Purpose}
Our purpose is to bring together motivated scholars of excellent academic ability and research potential together with excellent faculty in a vibrant research focussed environment that provides excellent research training and personal [career] development opportunities to produce outstanding doctoral scholars who are well trained and working in a stimulating research environment for a career in academia or research-based careers outside of academia. 

\section*{Values}
We aim to produce PGRs who are ambitious and aspire to be at the forefront of research in their area. Given the diversity of the student and academic body our PGRs would be well equipped to work in collaboration with other world leading international researchers. Our PGR activities will be embedded in all discipline and inter-disciplinary research activities with the school, enabling the creation of a vibrant community of PGRs who inspire and support one another, and who are an integral part of the school’s research community. Our PGRs would strive to generate impact not only academically but also to policy and practice outside of the university sector. Our PGRs would apply the highest standards of academic rigour and integrity in their work through critical, independent and creative thinking, systematic and methodological examination of the research problems and intellectual honesty underpinning their research.

\section*{Strategic Goals}
\begin{enumerate}
\item Attract and retain the highest quality students into our doctoral programme.
\item Provide excellent supervisory support, research training and personal development programmes that will train our PGRs to conduct rigorous, interesting, challenging, innovative and impactful research.
\item Give regular and ongoing administrative support for the timely completion and placement of PGRs. 
\item Revitalise the PGR provision to improve the structure and quality of PGR programme.
\end{enumerate}

\section*{Strategic Goals and Objectives}
\textbf{Goal 1: Attract and retain the highest quality students into our doctoral programme.}
\newline\textbf{Objectives}
\begin{enumerate}
\item Ensure the maximum visibility and reach of the available opportunities for doctoral research.
\item Admit highly motivated, academically strong candidates with excellent research potential. 
\item Ensure that our application process and management systems are fair, transparent, and efficient.
\item Grow our collaboration with academic and non-academic partners to raise our international profile and reputation.
\item Grow PGR numbers sustainably.
\end{enumerate}
\textbf{Goal 2: Provide excellent supervisory support, research training and personal development programmes that will train our PGRs to conduct rigorous, interesting, challenging, innovative and impactful research.}
\textbf{Objectives}
\begin{enumerate}
\item Promote and encourage the take up of supervisor development and training programs.
\item Ensure that the training we provide is of sufficient quality and rigour and provides a good training that equips PGRs to deal with the complexity and challenges of research that has impact in academia, policy and practice. 
\item Provide opportunities for personal and career development.
\item Ensure that PGRs are trained in aspects of research integrity and ethics and to uphold highest standards of intellectual honesty.
\item Promote a proactive attitude towards health and wellbeing. 
\item Ensure that PGRs are embedded in all discipline and interdisciplinary research activities.
\end{enumerate}
\textbf{Goal 3: Give regular and ongoing administrative support for the timely completion and placement of PGRs.}
\newline\textbf{Objectives}
\begin{enumerate}
\item Ensure that PhDs are complete within the maximum period of study allowed.
\item Provide support to PGRs in preparing for the job market.
\item Ensure that graduates are provided with opportunities to engage with and contribute to the doctoral programme and the school more generally.
\end{enumerate}
\textbf{Goal 4: Revitalise the PGR provision to improve the structure and quality of PGR programme.}
\newline\textbf{Objectives}
\begin{enumerate}
\item Review and updating of the PGR provision by proactively soliciting feedback from different stakeholders.
\item Benchmarking PGR activities to immediate peers and aspirational peers to ensure that our programme remains competitive.
\item Ensure the policies and procedures are consistent with the university policies and procedures as appropriate
\end{enumerate}

\section*{Action Plans}

\textbf{Goal 1: Attract and retain the highest quality students into our doctoral programme.}
\begin{enumerate}
	\item Ensure the maximum visibility and reach of available opportunities at the school for doctoral research.
	\begin{itemize}
		\item Participation in key education fairs and co-sponsoring targeted academic events such as the Academy of Management.
		\item Engaging with successful alumnus from particular geographical regions or institutions both for promotion and funding opportunities. 
		\item Maintain up to-date and easy to navigate and search webpages where we highlight the key research strengths (by discipline and cluster/centre), identify priority research areas (where we have critical mass in four star research), maintain a database of specific funded and unfunded doctoral research projects, areas of interest for PhD supervision and the availability of individual academics to supervise.
		\item Our webpages should highlight the interesting topics that are currently being explored by our scholars, profiles of current PGRs, highlight profiles of well-placed alumnus.
		\item Clarity on details of funded opportunities available and promote them well in advance.
		\item Where appropriate, create PhD programs that accurately reflect the specialism within the broad discipline, particularly within Management and in new research areas particularly those that are inter-disciplinary.
	\end{itemize}
\item Admit highly motivated, academically strong candidates with excellent research potential. 
\begin{itemize}
\item Entry criteria should be based on both academic achievements and research potential and motivation, require standardised test scores as a part of the entry requirement, research potential and to be assessed through  interviews/discussions with potential scholars, covering letters and to a smaller extent by references. 
\end{itemize}
\item Ensure that our application process and management systems are fair, transparent, and efficient.
\begin{itemize}
\item Clarity on the application process, particularly around initial contact with potential supervisors in relation to discussion of potential research topics within the area of interest (less relevant where the students come through to the PhD program through a rigorous in-house training program).
\item Agreed “in-process times” for various stages of the application to offer cycle, timely communication and update to the applicants in relation to progress of application and offers/rejections as appropriate. 
\item Ensure that the application system does not discriminate against any individuals or groups of individuals. 
\end{itemize}
\item Grow our collaboration with academic and non-academic partners to raise our international profile and reputation.
\begin{itemize}
\item Collaborate and partner with reputed international educational institutions for PhD exchanges, co-tutelles, co-supervision of PhDs and joint PhDs. 
\item Collaborate and partner with non-academic organisations for funding, explore opportunities to explore research questions of practical significance.
\end{itemize}
\item Grow PGR numbers sustainably.
\begin{itemize}
\item A key driver of PGR numbers to be virtuous cycle where reputation of the doctoral programme and superior placements of our PGRs drive demand for places (so goals 2, 3 and 4 feed into 1.)
\item Create additional funding opportunities of which a major part will be school committed funding for scholarships, supplemented by funding through alumni network, partnerships with industry partners (individual or consortia) and research and industry bodies (such as the UNPRI), enhanced training and support for increasing our share of RCUK funding, understanding the criteria for inclusion on government shortlists, predetermined and guaranteed funding but flexible funding models which are promotable on all of our marketing materials rather than contingent funding as far as possible.  
\item Fill supervisory capacity and slack taking into account - funding and publication pipelines, understand  supervision dynamics to enable mentoring of ECRs and junior faculty in the supervisor of PGRs by being part of more experienced teams, opportunities for junior faculty to participate in the PGR process through participation in the examination process. 
\item Where appropriate and feasible, encourage and support cross-disciplinary, cross-college or cross institutional supervisory teams.
\item Where appropriate, incentivise PGR supervision, build provision for PGR into grant applications where allowed and is feasible. 
\item Benchmark both against our immediate peers and aspirational peers in addition to the Russell group in terms of PGR/FTE numbers.
\end{itemize}
\end{enumerate}
\textbf{Goal 2: Provide excellent supervisory support, academic training and personal development programmes that will enable PGRS to conduct rigorous, interesting, challenging, innovative and impactful research.}
\begin{enumerate}
\item Promote and encourage the take up of supervisor development and training programs.
\begin{itemize}
\item Create effective policies and procedures that support supervisors to provide excellent supervisory support, manage expectations of both PGRs and supervisors.
\item Ensure the pastoral tutor system works as intended to provide an independent support system apart from the supervisory team. 
\end{itemize}
\item Ensure that the training we provide is of sufficient quality and rigour and provides a good training that equips PGRs to deal with the complexity and challenges of research that has impact in academia, policy and practice. 
\begin{itemize}
\item Rigorous training built into the PhD programme through alternative models such as a  2 + 3 PhD model, initially in economics but to eventually to all disciplines.  
\item The research training should give PGRs a sound grounding in both established theories and methods and also current developments in relation to both these in their discipline and area of research, training should cover both advanced quantitative methods and aspects of qualitative research with the balance between the two as appropriate to the discipline/area of research, a clear understanding of the state of research in their field and identify gaps.
\item Training should enable critical thinking, independent research, examination of complex problems of a multidisciplinary nature, appreciation of impact outside academia to policy and practice.
\item Training should also cover, especially for those aiming for an academic career, an exposure to the publishing process including targeting appropriate journals, distinguishing top quality research, referring process, preparing referee reports and responding to referee reports, training should also enable PGRS to show to structure a research paper, being a conference discussant and write/present to  non-specialist audiences.
\item Where appropriate, training should include training in databases and econometric/statistical software, bibliography and type setting software, emerging tools/methods for analysing big data - particularly open source tools and use of electronic library and other library resources.
\item Where appropriate and relevant, to allow PGRs to audit courses in other colleges/disciplines, identify and recommend courses available through platforms such as coursera, edX and other online sources. 
\item Work closely with internal partners to such as the doctoral college to promote and encourage the training offered including mandatory training, researcher development training, and support for writing and also to work with external partners like the GW4 to enable access to training they provide or indeed to pool resources in delivering training. 
\item Provide opportunities to gain valuable experience in teaching and learning, running help desks, running training sessions or workshops within established guidelines for PGRs so as to maintain a good balance with their studies. 
\end{itemize}
\item Provide opportunities for personal and career development.
\begin{itemize}
\item Career development opportunities in the form of opportunity to meet relevant external speakers and visitors, presentation and discussion in student led brown bag seminars, participation in discussion groups, participation in the annual student conference and the annual festival of doctoral research, exposure to researcher development framework (VITAE), opportunities to participate in workshops and seminars organised at the discipline research clusters/centres.
\item Promote and encourage the take up of training opportunities around personal development in the form of time-management skills, stress management, mindfulness among others. 
\item Provide opportunities to present at international conferences – every discipline should have a list of four top conferences in the discipline and if a student is accepted for presentation or discussion should be funded for one such opportunity. Provide opportunities to visit partner universities as part of student exchange programs or visiting scholar schemes where possible opportunities to fund these should be explored, but at least some of these should be funded, at least in part. 
\item Provide opportunities for exchange of research ideas, discussion of research related matters, sharing of information on research and other academic related events that can be accessed both by students on campus as well as part-timers and off-campus students. 
\end{itemize}
\item Ensure that PGRS are trained in aspects of research integrity and ethics and to uphold highest standards of intellectual honesty.
\begin{itemize}
\item Train PGRS in the ethics of research and in relation to all aspects of research, ensure familiarity with policies and procedures and good practice in the conduct of research. 
\end{itemize}
\item Promote a proactive attitude towards health and wellbeing. 
\begin{itemize}
\item Where appropriate Health Wellbeing and support for study procedures (HWSSP) to be applied in cases where physical or mental ill health or disability impairs academic progress and engagement, ensure that all involved in the supervisor and management of PGRs are familiar with the HWSSP processes and procedures and for this to be highlighted in any induction/training for supervisors and others involved in PGR activities. 
\item Ensure that pastoral support is adequate and pastoral tutors are trained to perform their role effectively, periodically review and make adjustments to the pastoral tutor provision as appropriate. 
\end{itemize}
\item Ensure that PGRs are embedded in all discipline and interdisciplinary research activities.
\begin{itemize}
	\item Where appropriate PGRs to be invited to participate in all cluster/center research activities.
\end{itemize}
\end{enumerate}
\textbf{Goal 3: Support the timely completion of the doctoral programme and placements of PGRs.}
\begin{enumerate}
\item Ensure that PhDs are complete within the maximum period of study allowed.
\begin{itemize}
	\item Set up, review and update policies and procedures that allow for better progress monitoring and assessment with clear criteria around the requirements for satisfactory progress and follow up, careful monitoring of movement to continuation status, appointment of examiners and managing the post examination process. 
\end{itemize}
\item Provide support to PGRs in preparing for the job market.
\begin{itemize}
	\item Promote where appropriate, opportunities for internships and collaboration with external partners. 
	\item Support for preparation for the job market in the form of training for writing academic CVs, conduct mock interviews, checking cover letters.
	\item Showcasing of job market candidates on PGR webpages, a placement officer to handle placements.
\end{itemize}
\item Ensure that graduates are provided with opportunities to engage with and contribute to the doctoral programme and the school more generally.
\begin{itemize}
	\item Track placement of our doctoral scholars, showcase placements and profiles of well-placed PGRs, create a doctoral alumni body and engage with doctoral alumni to contribute to the program and to the school through offering training, mentoring and other services.
\end{itemize}
\end{enumerate}
\textbf{Goal 4: Review and update the PGR provision to improve the structure and quality of the doctoral programme.}
\begin{enumerate}
\item Review and updating of the PGR provision by proactively soliciting feedback from different stakeholders.
\begin{itemize}
\item Where feasible and appropriate, ensure that discussion of PGR matters  at the school, department, research cluster/center.
\end{itemize}
\item Where appropriate, benchmarking PGR programme and training provision to our immediate peers and aspirational peers to ensure that our programme remains competitive.
\begin{itemize}
	\item Periodically review and update the structure of the program to allow us to match our immediate peers and to improve our PGR Programme.
\end{itemize}
\item Ensure the policies and procedures are consistent with the university policies and procedures as appropriate.
\begin{itemize}
	\item Periodically review and revise our policies and procedures to ensure our policies, procedures and practice  are consistent with the university regulations 
\end{itemize}
\end{enumerate}

\section*{Goals, Objectives and KPIs}
\textbf{Goal 1: Attract to and retain in our doctoral programme, high calibre PGRs.}
\textbf{Objectives: see above}
\textbf{KPIs}
\begin{itemize}
\item Measures of quality of incoming doctoral scholars – academic performance, past research experience, funded through competitions or other prestigious scholarships
\item Reputation of external Collaborators and Partners and the nature of collaboration.
\item Demographic diversity of applicants
\item Number of externally funded places.
\item Applications to (target) places ratio.
\item Offer to acceptance ratio.
\item PGR/Staff (FTE)
\item Processing times for applications. 
\end{itemize}
\textbf{Goal 2: Provide supervisory support, academic training and personal development programmes that will enable our PGRS to conduct rigorous, interesting, challenging, innovative and impactful research.}
\textbf{Objectives: see above}
\textbf{KPIs}
\begin{itemize}
\item PRES scores
\item Number and type of Research related training activities
\item Number and type  of career personal development activities
\item Number of Student exchanges and visits.
\item Number of Conference participation (presentations and/or discussion) by PGRs.
\item Number of Complaints/Appeals and resolutions.
\item Hardship support.
\item Upgrades – MRes to Mphil, Mphil to PhD – timing and success rates.
\item Exit and withdrawals.
\item Health and wellbeing concerns raised versus application of Health and well-being study procedures.
\end{itemize}
\textbf{Goal 3: Support the timely completion of the doctoral programme and placements of PGRs}
\textbf{Objectives: see above}
\textbf{KPIs}
\begin{itemize}
\item Completion times.
\item Time to appointment of panel of examiners from final submission.
\item Number of Extensions and Interruptions.
\item Number of type of placement support activities.
\item Number and type of Alumni engagements. 
\item Destination of PGRs – academic – Russell group, others, Non-academic- research think tanks, others?
\end{itemize}
\textbf{Goal 4: Solicit and act on feedback and input from the PGRs and other stakeholders in continuously improving our PGR programme.}
\textbf{Objectives: see above}
\textbf{KPIs}
\end{document}



